\thusetup{
  %******************************
  % 注意:
  %   1. 配置里面不要出现空行
  %   2. 不需要的配置信息可以删除
  %******************************
  %
  % 中国海洋大学研究生学位论文封面
  % 参考:中国海洋大学研究生学位论文书写格式20130307.doc
  % 为避免出现错误,下面保留[清华大学学位论文模板原有定义无需修改],
  % 请直接跳到后面[中国海洋大学学位论文模板部分请根据自己情况修改]。
  %
%%%%%%%%%%%%%%%%%%%%%%[清华大学学位论文模板原有定义无需修改]%%%%%%%%%%%%%%%%%%%%%%%
  %=====
  % 秘级
  %=====
  secretlevel={秘密},
  secretyear={10},
  %
  %=========
  % 中文信息
  %=========
  ctitle={清华大学学位论文 \LaTeX\ 模板\\使用示例文档 v\version},
  cdegree={工学硕士},
  cdepartment={计算机科学与技术系},
  cmajor={计算机科学与技术},
  cauthor={薛瑞尼},
  csupervisor={郑纬民教授},
  cassosupervisor={陈文光教授}, % 副指导老师
  ccosupervisor={某某某教授}, % 联合指导老师
  % 日期自动使用当前时间,若需指定按如下方式修改:
  % cdate={超新星纪元},
  %
  % 博士后专有部分
  cfirstdiscipline={计算机科学与技术},
  cseconddiscipline={系统结构},
  postdoctordate={2009年7月——2011年7月},
  id={编号}, % 可以留空: id={},
  udc={UDC}, % 可以留空
  catalognumber={分类号}, % 可以留空
  %
  %=========
  % 英文信息
  %=========
  etitle={An Introduction to \LaTeX{} Thesis Template of Tsinghua University v\version},
  % 这块比较复杂,需要分情况讨论:
  % 1. 学术型硕士
  %    edegree:必须为Master of Arts或Master of Science(注意大小写)
  %             “哲学、文学、历史学、法学、教育学、艺术学门类,公共管理学科
  %              填写Master of Arts,其它填写Master of Science”
  %    emajor:“获得一级学科授权的学科填写一级学科名称,其它填写二级学科名称”
  % 2. 专业型硕士
  %    edegree:“填写专业学位英文名称全称”
  %    emajor:“工程硕士填写工程领域,其它专业学位不填写此项”
  % 3. 学术型博士
  %    edegree:Doctor of Philosophy(注意大小写)
  %    emajor:“获得一级学科授权的学科填写一级学科名称,其它填写二级学科名称”
  % 4. 专业型博士
  %    edegree:“填写专业学位英文名称全称”
  %    emajor:不填写此项
  edegree={Doctor of Engineering},
  emajor={Computer Science and Technology},
  eauthor={Xue Ruini},
  esupervisor={Professor Zheng Weimin},
  eassosupervisor={Chen Wenguang},
  % 日期自动生成,若需指定按如下方式修改:
  % edate={December, 2005}
  %
  % 关键词用“英文逗号”分割
  ckeywords={\TeX, \LaTeX, CJK, 模板, 论文},
  ekeywords={\TeX, \LaTeX, CJK, template, thesis}
}

% 定义中英文摘要和关键字
\begin{cabstract}
  论文的摘要是对论文研究内容和成果的高度概括。摘要应对论文所研究的问题及其研究目
  的进行描述,对研究方法和过程进行简单介绍,对研究成果和所得结论进行概括。摘要应
  具有独立性和自明性,其内容应包含与论文全文同等量的主要信息。使读者即使不阅读全
  文,通过摘要就能了解论文的总体内容和主要成果。

  论文摘要的书写应力求精确、简明。切忌写成对论文书写内容进行提要的形式,尤其要避
  免“第 1 章……;第 2 章……;……”这种或类似的陈述方式。

  本文介绍清华大学论文模板 \thuthesis{} 的使用方法。本模板符合学校的本科、硕士、
  博士论文格式要求。

  本文的创新点主要有:
  \begin{itemize}
    \item 用例子来解释模板的使用方法;
    \item 用废话来填充无关紧要的部分;
    \item 一边学习摸索一边编写新代码。
  \end{itemize}

  关键词是为了文献标引工作、用以表示全文主要内容信息的单词或术语。关键词不超过 5
  个,每个关键词中间用分号分隔。(模板作者注:关键词分隔符不用考虑,模板会自动处
  理。英文关键词同理。)
\end{cabstract}

% 如果习惯关键字跟在摘要文字后面,可以用直接命令来设置,如下:
% \ckeywords{\TeX, \LaTeX, CJK, 模板, 论文}

\begin{eabstract}
   An abstract of a dissertation is a summary and extraction of research work
   and contributions. Included in an abstract should be description of research
   topic and research objective, brief introduction to methodology and research
   process, and summarization of conclusion and contributions of the
   research. An abstract should be characterized by independence and clarity and
   carry identical information with the dissertation. It should be such that the
   general idea and major contributions of the dissertation are conveyed without
   reading the dissertation.

   An abstract should be concise and to the point. It is a misunderstanding to
   make an abstract an outline of the dissertation and words ``the first
   chapter'', ``the second chapter'' and the like should be avoided in the
   abstract.

   Key words are terms used in a dissertation for indexing, reflecting core
   information of the dissertation. An abstract may contain a maximum of 5 key
   words, with semi-colons used in between to separate one another.
\end{eabstract}

% \ekeywords{\TeX, \LaTeX, CJK, template, thesis}
%%%%%%%%%%%%%%%%%%%%%%%%%%%%%%%%%%%%%%%%%%%%%%%%%%%%%%%%%%%%%%%%%%%%%%%%%%%%%%%%

%%%%%%%%%%%%%%%%%%[中国海洋大学学位论文模板部分请根据自己情况修改]%%%%%%%%%%%%%%%%%%%
% 中国海洋大学研究生学位论文封面
% 必须填写的内容包括(其他最好不要修改):
%   分类号、密级、UDC
%   论文中文题目、作者中文姓名
%   论文答辩时间
%   封面感谢语
%   论文英文题目
%   中文摘要、中文关键词
%   英文摘要、英文关键词
%
%%%%%[自定义]%%%%%
\newcommand{\fenleihao}{}%分类号
\newcommand{\miji}{}%密级 
                    % 绝密$\bigstar$20年 
                    % 机密$\bigstar$10年
                    % 秘密$\bigstar$5年
\newcommand{\UDC}{}%UDC
\newcommand{\oucctitle}{论文题目}%论文中文题目
\ctitle{论文题目}%必须修改因为页眉中用到
\cauthor{XXX}%可以选择修改因为仅在 pdf 文档信息中用到
\cdegree{工学博士}%可以选择修改因为仅在 pdf 文档信息中用到
\ckeywords{\TeX, \LaTeX, CJK, 模板, 论文}%可以选择修改因为仅在 pdf 文档信息中用到
\newcommand{\ouccauthor}{牛文杰}%作者中文姓名
%\newcommand{\ouccsupervisor}{姬光荣教授}%作者导师中文姓名
%\newcommand{\ouccdegree}{博\hspace{1em}士}%作者申请学位级别
%\newcommand{\ouccmajor}{海洋信息探测与处理}%作者专业名称
%\newcommand{\ouccdateday}{\CJKdigits{\the\year}年\CJKnumber{\the\month}月\CJKnumber{\the\day}日}
%\newcommand{\ouccdate}{\CJKdigits{\the\year}年\CJKnumber{\the\month}月}
\newcommand{\oucdatedefense}{                }%论文答辩时间
%\newcommand{\oucdatedegree}{2009年6月}%学位授予时间
\newcommand{\oucgratitude}{谨以此论文献给我的导师和亲人!}%封面感谢语
\newcommand{\oucetitle}{English Title}%论文英文题目
%\newcommand{\ouceauthor}{Haiyong Zheng}%作者英文姓名
\newcommand{\oucthesis}{\textsc{OUCThesis}}
%%%%%默认自定义命令%%%%%
% 空下划线定义
\newcommand{\oucblankunderline}[1]{\rule[-2pt]{#1}{.7pt}}
\newcommand{\oucunderline}[2]{\underline{\hskip #1 #2 \hskip#1}}

% 论文封面第一页
%%不需要改动%%
\vspace*{5cm}
{\xiaoer\heiti\oucgratitude

\begin{flushright}
---\hspace*{-2mm}---\hspace*{-2mm}---\hspace*{-2mm}---\hspace*{-2mm}---\hspace*{-2mm}---\hspace*{-2mm}---\hspace*{-2mm}---\hspace*{-2mm}---\hspace*{-2mm}---~\ouccauthor
\end{flushright}
}

\newpage

% 论文封面第二页
%%不需要改动%%
\vspace*{1cm}
\begin{center}
  {\xiaoer\heiti\oucctitle}
\end{center}
\vspace{10.7cm}
{\normalsize\songti
\begin{flushright}
{\renewcommand{\arraystretch}{1.3}
  \begin{tabular}{r@{}l}
    学位论文答辩日期:~ & \oucunderline{1.8em}{\oucdatedefense} \\
    指导教师签字:~ & \oucblankunderline{5cm} \\
    答辩委员会成员签字:~ & \oucblankunderline{5cm} \\
    ~ & \oucblankunderline{5cm} \\
    ~ & \oucblankunderline{5cm} \\
    ~ & \oucblankunderline{5cm} \\
    ~ & \oucblankunderline{5cm} \\
    ~ & \oucblankunderline{5cm} \\
    ~ & \oucblankunderline{5cm} \\
  \end{tabular}
}
\end{flushright}
}

\newpage

% 论文封面第三页
%%不需要改动%%
\vspace*{1cm}
\begin{center}
  {\xiaosan\heiti 独\hspace{1em}创\hspace{1em}声\hspace{1em}明}
\end{center}
\par{\normalsize\songti\parindent2em
本人声明所呈交的学位论文是本人在导师指导下进行的研究工作及取得的研究成果。据我所知,除了文中特别加以标注和致谢的地方外,论文中不包含其他人已经发表或撰写过的研究成果,也不包含未获得~\oucblankunderline{7cm}(注:如没有其他需要特别声明的,本栏可空)或其他教育机构的学位或证书使用过的材料。与我一同工作的同志对本研究所做的任何贡献均已在论文中作了明确的说明并表示谢意。
}
\vskip1.5cm
\begin{flushright}{\normalsize\songti
  学位论文作者签名:\hskip2cm 签字日期:\hskip1cm 年 \hskip0.7cm 月\hskip0.7cm 日}
\end{flushright}
\vskip.5cm
{\setlength{\unitlength}{0.1\textwidth}
  \begin{picture}(10, 0.1)
    \multiput(0,0)(0.2, 0){50}{\rule{0.15\unitlength}{.5pt}}
  \end{picture}}
\vskip1cm
\begin{center}
  {\xiaosan\heiti 学位论文版权使用授权书}
\end{center}
\par{\normalsize\songti\parindent2em
本学位论文作者完全了解学校有关保留、使用学位论文的规定,并同意以下事项:
\begin{enumerate}
\item 学校有权保留并向国家有关部门或机构送交论文的复印件和磁盘,允许论文被查阅和借阅。
\item 学校可以将学位论文的全部或部分内容编入有关数据库进行检索,可以采用影印、缩印或扫描等复制手段保存、汇编学位论文。同时授权清华大学“中国学术期刊(光盘版)电子杂志社”用于出版和编入CNKI《中国知识资源总库》,授权中国科学技术信息研究所将本学位论文收录到《中国学位论文全文数据库》。
\end{enumerate}
(保密的学位论文在解密后适用本授权书)
}
\vskip1.5cm
{\parindent0pt\normalsize\songti
学位论文作者签名:\hskip4.2cm\relax%
导师签字:\relax\hspace*{1.2cm}\\
签字日期:\hskip1cm 年\hskip0.7cm 月\hskip0.7cm 日\relax\hfill%
签字日期:\hskip1cm 年\hskip0.7cm 月\hskip0.7cm 日\relax\hspace*{1.2cm}}

\newpage

\pagestyle{plain}
\clearpage\pagenumbering{roman}

% 中文摘要
%%[需要填写:中文摘要、中文关键词]%%
\begin{center}
  {\sanhao[1.5]\heiti\oucctitle\\\vskip7pt 摘\hspace{1em}要}
\end{center}
{\normalsize\songti

习近平同志在党的十九大报告中指出:“坚持陆海统筹,加快建设海洋强国。”水下视觉是海洋研究的基础,在建设海洋强国的今天,水下视觉研究具有重要现实意义、战略意义。由于技术和设备的局限性,水下数据采集耗费巨大、成本高昂,海洋环境中获取特定要求的图像数据困难重重;另一方面由于水下光学物理特性,光在水介质中的传播过程存在吸收和散射,会出现色偏、模糊、有雾状效果等多种样式,却很难获取到指定样式的水下数据。但是目前许多水下图像处理任务,都需要大量成对以及配对的多种样式水下数据。随着深度学习的不断发展,生成对抗网络开始越来越多地被应用于水下视觉研究,逐步为水下图像处理提供了简洁、有效、不受物理模型限制、具有泛化性的方法。

基于生成对抗网络的图像转译方法在目前水下图像合成中占据重要的地位,通过了解现有基于生成对抗网络的水下图像多样式转译问题的国内外现状发现当前的方法还存在缺陷,本工作针对水下图像合成方法无法同时实现多样式的转译问题进行了研究和解决。本文主要对水下图像的多样式合成问题进行了深入的探讨,通过对图像转译工作进行研究,在图像转译工作的多模态转译和多域转译方向开展实验,提出两种不同思路的模型进行解决。本文主要工作如下:

第一,针对水下图像多样式合成的问题,我们研究了基于生成对抗网络的图像转译领域,在充分了解水下图像合成工作后进一步梳理出水下图像多样式合成可行的两种思路并详尽分析。

第二,一种设计思路是基于跨域图像多模态转译方法,将多个水下样式看作水下域的多个模态进行处理,提出的水下图像跨域多模态转译模型引入内容一致性损失限制内容在转译过程中保持完整不损耗,并对该思路进行具体的网络模型结构设计和目标函数设置。

第三,另一种思路是基于多域转译的思想进行设计,将包括空中图像在内的各个样式视为多个样式域,把多样式转译问题建模于多域转译,通过样式编码控制生成器转译到各个样式域,并对该设计思路了搭建了具体的网络模型结构和设置了目标函数。

最后,对于设计的水下图像多样式转译的两种思路,对比图像转译问题中最经典和最前沿的方法,在多个合成数据集和真实水下数据集上设计了一系列对比实验和消融实验。对每组实验结果进行了详细的评价和分析,定性视觉结果和定量指标结果都验证了我们提出的网络结构可以有效解决水下图像多样式转译问题,以及得到比现有其它方法更好的结果。
}
\vskip12bp
{\xiaosi\heiti\noindent
关键词:生成对抗网络,图像到图像转译,多模态转译,多域转译,水下图像合成}

\newpage

% 英文摘要
%%[需要填写:英文摘要、英文关键词]%%
\begin{center}
  {\sanhao[1.5]\heiti\oucetitle\\\vskip7pt Abstract}
\end{center}
{\normalsize\songti

Xi Jinping pointed out in the report of the 19th National Congress of the Communist Party of China: Adhere to the overall planning of land and sea, and speed up the construction of a maritime power. Underwater vision is the foundation of marine research. In the time of building a maritime power, underwater vision research has important practical significance and strategic significance. Due to the limitations of technology and equipment, underwater data collection is expensive and costly, and it is difficult to obtain specific required image data in the marine environment. On the other hand, due to the underwater optical and physical characteristics, there is absorption and scattering in the process of light propagation in the water medium. There will be a variety of patterns such as color cast, blur, and foggy effects, but it is difficult to obtain underwater data of the specified styles. However, many underwater image processing tasks currently require a large number of pairs and pairs of underwater data in multiple styles. With the continuous development of deep learning, Generative Adversarial Networks have begun to be increasingly used in underwater vision research, and gradually provide a concise, effective, and generalized method for underwater image processing that is not restricted by physical models.

The image translation method based on the generative adversarial network occupies an important position in the current underwater image synthesis. By understanding the domestic and foreign situation of the existing underwater image multi-style translation based on the generative adversarial network, it is found that the current method still has shortcomings. This work research and solve the problem that underwater image synthesis methods cannot achieve multi-style translation at the same time. This paper mainly discusses the problem of multi-style synthesis of underwater images. Through the research of image translation, experiments are carried out in the direction of multi-modal translation and multi-domain translation, and two models of different ideas are proposed. solve. The main work of this paper is as follows:

First, for the problem of underwater image multi-style synthesis, we studied the field of image translation based on generative adversarial networks. After fully understanding the underwater image synthesis work, we further work out two possible ideas for underwater image multi-style synthesis and analyzed them in detail.

Second, a design idea is based on a cross-domain image multi-modal translation method, which treats multiple underwater styles as multiple modals in the underwater domain for processing, and the proposed underwater image cross-domain multi-modal translation model is introduced the content-consistency loss as the limitation to remain the content intact and not lost during the translation process, and the corresponding network structure design and the function of objective are carried out later.

Third, a design is based on the idea of multi-domain translation, which treats each style including in-air images as multiple style domains, the problem of multi-style are translated to multi-domain translation, and control the generation through style codes. The generator translates the input into various style domains, and a corresponding network structure and objective are set up for the design idea.

Finally, for the two ideas of designing underwater image multi-style translation, comparing the most classic and cutting-edge methods in the image translation problem, a series of comparative experiments and experiments were designed on multiple synthetic data sets and real underwater data sets. Ablation experiment. The qualitative visual results and the quantitative index results verify that the network structure proposed by us can effectively solve this problem and its superiority. 

Finally, for the two ideas of designing underwater image multi-style translation, comparing the most classic and the state-of-the-art methods in the image translation problem, a series of comparative experiments and ablation studies were designed on multiple synthetic and real-world underwater datasets. Both qualitative visual results and quantitative metrics results verify that our proposed network can effectively solve the problem of underwater image multi-style translation and obtain better results than other existing methods.

}
 
\vskip12bp
{\xiaosi\heiti\noindent 
\textbf{Keywords: Generative Adversarial Network, Image-to-image Translation, Multi-modal Translation, Multi-domain Translation, Underwater Image Synthesis}}
%%%%%%%%%%%%%%%%%%%%%%%%%%%%%%%%%%%%%%%%%%%%%%%%%%%%%%%%%%%%%%%%%%%%%%%%%%%%%%%%
