\chapter{绪论}
\section{研究背景和意义}
% 图像合成问题起源 定义 分类有哪些 任务应用 发展趋势
计算机视觉领域中,数据是解决一切问题的基础;缺乏海量的语音、图像、视频等数据来进行研究,是现在面临的最大挑战。针对匮乏的数据,具有真实性的数据生成问题一直是遏制计算机视觉发展和亟需解决的问题。从生成对抗网络问世开始,计算机视觉领域打开了一个全新的局面,愈来愈多的工作基于生成对抗网络开展。在图像处理领域,基于生成对抗网络,可以通过训练简单的对抗模型合成大量现实中不存在但是具有真实性的图像数据。

在生成对抗网络快速发展的过程中,图像到图像转译被单独划分成一类问题针对进行解决。原始的生成对抗网络图像转译是指将输入图像“转译”成一副相应的输出图像,输入图像与输出图像之间具有相同的内容但是具有不同的场景特征。类似我们谷歌翻译一句话,翻译器不会改变我们输入的内容和意思,但是可以翻译成英语、法语等其他语种。图像到图像的转译简称为图像转译,目的是将一个图像从当前场景转译到其他场景中,比如转译成一副对应内容的边缘信息图、语义标签图或者特定场景风格图。我们将这种转译视为从源域映射到目标域的转译,输入和输出分别属于两个不同的域。

在研究图像转译问题中,根据可获得的数据集是否包含成对的训练图像可以将解决方法划分为成对的图像转译子问题和非成对的图像转译子问题。当数据集满足包含成对的训练图像时,成对的图像转译问题就可以通过条件生成对抗网络等有监督的方法来进行解决。现实情况更多的是不满足获取到大量的成对数据,或者像水下场景等获取成对训练数据特别困难、代价高昂,非成对图像数据的无监督转译模型被提出。另外,源域和目标域的训练图像数量不平衡,往往一个域中图像数据数量远远超过另一个域图像数据数量时给图像转译带来巨大的难度。

类似谷歌翻译器,不同语种之间的翻译很多情况下并不是一一对应的,往往一个语义内容可以对应多个翻译结果。在图像转译中,需要成对训练数据的有监督图像转译往往是一对一的转译,给定源域输入有真实的目标域生成图像数据;不成对训练数据的无监督图像转译可以进行一对多的转译,网络无法固定学习到特定的映射,因此可以产生多样输出。这个现象说明,一方面无法直接得到特定的输出这方面有缺陷,另一方面意味着我们可以使用一些手段得到多样化的输出结果,实现多模态输出。

图像到图像的转译,目前已经有大量的应用。现实中的很多问题都可以转换成图像转译问题进行解决,图像转译具有强大的应用价值。可以应用于边框图上色、素描画上色等任务,极大的解放了人力;可以应用于人脸表情漫画化、人体姿态图像编辑等,实现特定风格的图像生成和控制,已经应用于图像美化等商业化软件中;还可以应用于图像超分辨重建任务,提升图像质量;以及在医学图像、水下图像等多种特定学科的常规手段难以直接进行解决的问题yo中得到应用,给其他学科的图像问题解决提供了新的思路和方法。

在水下环境,一方面数据采集耗费巨大、成本昂贵,另一方面由于水介质存在光的吸收和散射,导致很难获取到大量高质量的清晰水下图像数据。进行水下图像研究时,以水下图像增强和复原为例,许多方法都先通过图像转译模型合成成对的水下图像数据,再进行水下图像的增强或者复原操作。可见,同时合成多种水下环境条件这些模态的高质量图像数据对水下图像研究领域意义重大,可以为水下视觉领域提供大量的训练数据,推动水下研究工作的发展和进步。随着深度学习的广泛发展,生成对抗网络开始越来越多的


\section{国内外研究现状}
\subsection{生成对抗网络研究现状}
Goodfellow在2014年提出生成对抗网络(Generative Adversarial Network, GAN)~\cite{goodfellow2014generative}后,计算机视觉领域开始乘风向上。生成对抗网络由生成器和判别器两个神经网络组成,生成器试图生成可以欺骗过判别器的真实性结果,判别器负责区分输入的样本是真实样本还是生成器生成的结果。这样处于对抗关系中的两个神经网络彼此不断优化,使得该网络的合成结果能够生成超越变分自动编码器(VAE)~\cite{kingma2013auto}、自回归模型(AR模型)等以前生成模型。越来越多的科研人员开始深入研究生成对抗网络,近年深度卷积生成对抗网络(DCGAN)、条件生成对抗网络(CGAN)等许多基础生成对抗网络的变种工作成功在文本到图像生成、图像到图像转译、视频预测等领域应用都发挥了独特而强大的功能。然而,生成对抗网络也有其缺点,譬如存在训练不稳定、会产生模式崩塌等问题,后续的大量变种工作针对这些问题在生成对抗模型上进行解决。


\subsection{图像转译研究现状}
基于生成对抗网络的图像到图像的转译(Image-to-Image Translation)首次被Isola等在2017年的工作pix2pix(Image-to-Image Translation with Conditional Adversarial Networks,pix2pix)~\cite{isola2017image}中提出,依托条件为图像的条件生成对抗网络的思想,训练图像使用尺寸相同的成对数据,是最典型的有监督图像转译方法。同年,朱俊彦等提出cycleGAN(Unpaired Image-to-Image Translation using Cycle-Consistent Adversarial Networks,cycleGAN)~\cite{zhu2017unpaired},基于生成对抗网络提出的循环一致性损失,该方法解决了更普遍没有成对训练数据的无监督图像转译问题。自此,图像转译开始被应用于各种具体问题的解决。

无监督图像转译方法中,UNIT~\cite{liu2017unsupervised}将问题定义----实现两个域之间的转译是满足两个域的联合分布,假设输入到生成器的潜在编码共享,通过编码器将两个域的图像数据编码到潜在空间中。收分解表达学习的影响,MUNIT~\cite{huang2018multimodal}在UNIT提出共享潜在编码的基础上假设潜在编码包括内容编码和风格编码,可以合成相同内容不同风格的多模态结果。DRIT~\cite{lee2018diverse}与MUNIT思路一致,在于内容编码和风格编码连接方式上有所区别。以上基于分解的图像转译方法广泛应用于风格转换和多种多模态图像转译任务中。

以上的图像转译方法都是在两个域之间,若要进行多个域之间的转译需训练多个域间生成器一一对应,StarGAN~\cite{choi2018stargan}用一个生成器结构学习多个域之间的映射从而实现多个域的转译。StarGAN v2~\cite{choi2020stargan}将StarGAN多域转译工作拓展至多模态,可以 实现多域多模态的图像转译任务。在DRIT工作基础上,DRIT++~\cite{lee2020drit++}被提出,将两个域之间的转译也拓展至多个域。


\subsection{水下图像合成研究现状} 
目前水下图像转译


\section{课题来源}
国家自然科学基金面上项目“ 类别不平衡条件下海洋浮游生物图像精细识别及其原位应用研究”(批准号:61771440)、   国家自然科学基金面上项目“ 海洋中小型浮游生物原位光学观测关键技术研究”(批准号:41776113)。


\section{论文内容和安排}
本工作主要内容和章节安排如下:

第一章为绪论部分,该部分主要针对基于生成对抗网络的图像转译和水下图像转译的研究背景和意义、国内外研究现状以及课题来源进行论述。

第二章介绍了生成对抗网络的基本概念、基于生成对抗网络的主流算法和基于生成对抗网络应用于图像转译的方法。详细对算法概念和原理、网络模型结构、生成对抗网络及主流算法优缺点以及在图像转译场景上进行了具体介绍。

第三章介绍了水下图像研究。

第四章主要描述了本工作提出的基于生成对抗网络的图像多模态转译模型,随算法流程、网络模型设计思路以及具体的模型实现细节做了详细展现。

第五章介绍了。

第六章对本工作进行贡献总结和讨论,分析研究问题的特点以及提出方法的优点和缺点,对研究过程中固有问题进行分析和展望。