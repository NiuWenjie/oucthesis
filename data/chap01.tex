\chapter{绪论}
\section{研究背景和意义}

计算机视觉领域中,数据是解决一切问题的基础;具有真实性数据的生成问题一直是亟需解决的问题。从生成对抗网络问世开始,计算机视觉领域打开了一个全新的局面,愈来愈多的工作基于生成对抗网络开展。在图像处理领域,基于生成对抗网络的工作蓬勃发展,可以通过训练简单的对抗模型合成大量现实中无法获取到但是具有足够真实性的图像数据。

在生成对抗网络快速发展的过程中,图像到图像转译被单独划分成一类问题进行解决。图像到图像的转译简称为图像转译,图像转译是指将输入图像“转译”成一副相应的输出图像,输入图像与输出图像之间具有相同的内容但是具有不同的场景特征。类似我们谷歌翻译一句话,翻译器不会改变我们输入的内容和意思,但是可以翻译成英语、法语等其它语种。比如转译成一副对应内容的边缘信息图、语义标签图或者特定场景风格图。我们将这种转译视为从源域映射到目标域的转译,输入和输出分别属于两个不同的域。

在研究图像转译问题中,根据可获得的数据集是否包含成对的训练图像可以将解决方法划分为成对的图像转译子问题和非成对的图像转译子问题。当数据集满足包含成对的训练图像时,成对的图像转译问题就可以通过条件生成对抗网络等有监督的方法来进行解决。现实情况更多的是不满足获取到大量的成对数据,或者像水下场景等获取成对训练数据特别困难、代价高昂,非成对图像数据的无监督转译模型被提出。另外,源域和目标域的训练图像数量不平衡,往往一个域中图像数据数量远远超过另一个域图像数据数量,此时给图像转译带来巨大的难度。

类似谷歌翻译器,不同语种之间的翻译很多情况下并不是一一对应的,一个语义内容可以对应多个翻译结果。在图像转译中,需要成对训练数据的有监督图像转译是一对一的转译,源域输入有对应的目标域真实图像数据;在不成对数据上训练的无监督图像转译模型可以进行一对多的转译,网络无法固定学习到特定的映射,可以产生多样输出即多模态转译。从一个语种到多个语种的翻译,是同一个语意内容到不同语言环境下的表达。类似地,在图像转译中,当目标属于多个种类时,目标域就是多个域,从源域到多个目标域的转译称为多域转译。多模态转译和多域转译工作是图像转译的两个经典方向。

图像到图像的转译,目前已经有大量的应用。现实中的很多问题都可以转换成图像转译问题进行解决,图像转译具有强大的应用价值。可以应用于边框图上色、素描画上色等任务,极大的解放了人力;可以应用于图像超分辨重建任务,提升图像质量;可以应用于人脸表情漫画化、人体姿态图像编辑等,实现特定风格的图像生成和控制,已经落地图像美化等商业软件中;以及在医学图像、水下图像等多种特定学科的常规手段难以直接进行解决的问题中得到应用,给其他学科的图像问题解决提供了新的思路和方法。

在水下环境中,由于采集技术和设备的局限性,真实水下环境中的数据采集耗费巨大、成本高昂;另一方面由于水下光学物理特性,光在水介质的传播过程中有吸收和散射,会出现色偏、模糊、有雾状效果等多种样式,却很难获取到大量多样式水下图像数据。许多水下图像处理任务,都需要成对以及多种样式的水下数据。可见,同时合成多种水下环境条件的高质量图像数据对水下图像研究领域意义重大,可以为水下视觉领域提供大量的训练数据,推动水下研究工作的发展和进步。随着深度学习的广泛发展,生成对抗网络开始越来越多的被应用于水下图像研究,为水下图像处理提供了简洁、有效、不受物理模型限制、具有泛化性的方法。


\section{国内外研究现状}
\subsection{生成对抗网络研究现状}
Goodfellow在2014年提出生成对抗网络(Generative Adversarial Network, GAN)~\cite{goodfellow2014generative}后,计算机视觉领域开始乘风向上。生成对抗网络由生成器和判别器两个神经网络组成,生成器试图生成可以欺骗过判别器的真实性结果,判别器负责区分输入的样本是真实样本还是生成器生成的结果。这样处于对抗关系中的两个神经网络彼此不断优化,使得该网络的合成结果能够超越变分自动编码器(VAE)~\cite{kingma2013auto}、自回归模型(AR模型)等以前生成模型的结果。越来越多的科研人员开始深入研究生成对抗网络,近年深度卷积生成对抗网络(DCGAN~\cite{radford2015unsupervised})、条件生成对抗网络(CGAN~\cite{mirza2014conditional})等许多基础生成对抗网络的变种工作成功在文本到图像转译、图像到图像转译、视频预测等领域应用都发挥了独特而强大的功能。然而,生成对抗网络也有其缺点,譬如存在训练不稳定、会产生模式崩塌等问题,后续的大量变种工作针对这些问题在生成对抗模型上进行解决。


\subsection{图像转译研究现状}
基于生成对抗网络的图像到图像的转译(Image-to-image Translation)首次被Isola等人在2017年的工作pix2pix(Image-to-image Translation with Conditional Adversarial Networks,pix2pix)~\cite{isola2017image}中提出,依托条件为图像的条件生成对抗网络思想,训练图像使用尺寸相同的成对数据,是最典型的有监督图像转译方法。同年,Zhu等人提出CycleGAN(Unpaired Image-to-Image Translation using Cycle-Consistent Adversarial Networks,cycleGAN)~\cite{zhu2017unpaired},基于生成对抗网络提出的循环一致性损失,该方法解决了更普遍没有成对训练数据的无监督图像转译问题。自此,图像转译开始被应用于各种具体问题的解决。

无监督图像转译方法中,UNIT~\cite{liu2017unsupervised}将问题定义为实现两个域之间的转译是满足两个域的联合分布,假设输入到生成器的潜在编码共享,通过编码器将两个域的图像数据编码到潜在空间中。收受分解表达学习的影响,MUNIT~\cite{huang2018multimodal}在UNIT提出共享潜在编码的基础上假设潜在编码包括内容编码和风格编码,可以合成内容相同风格不同的多模态结果。DRIT~\cite{lee2018diverse}与MUNIT思路一致,仅在内容编码和风格编码连接方式上有所区别。以上基于分解表示的图像转译方法广泛应用于风格迁移和多种多模态图像转译任务中。DSMAP~\cite{chang2020domain}在MUNIT基础上利用特定于域的映射将共享内容空间中的潜在特征重新映射于特定于域的内容空间,这样保证特定于域的内容空间不含有源域的任何信息。

以上方法都是在两个域之间做图像转译,若要进行多个域之间的转译需训练多个域间生成器一一对应,StarGAN~\cite{choi2018stargan}用一个生成器结构学习多个域之间的映射从而实现多个域的转译。StarGAN v2~\cite{choi2020stargan}将StarGAN多域转译工作拓展至多模态,可以 实现多域多模态的图像转译任务。在DRIT工作基础上,DRIT++~\cite{lee2020drit++}被提出,将两个域之间的转译也拓展至多个域。


\subsection{水下图像合成研究现状} 
水下图像合成用于水下图像增强和水下图像复原等各种水下图像处理任务。除了基于光学的物理模型进行图像处理外还有基于深度学习的水下图像处理方法。基于生成对抗网络的水下图像转译我们也称作水下图像合成。基于深度学习的水下图像合成,目前主要有两种常用的方式。

第一种是基于物理光学信息,主要在室内图像数据集上应用,其中需要额外的深度图或者传输图等水下信息来生成具有特定水下条件特征的合成图像。WaterGAN~\cite{li2017watergan}使用RGB-D~\cite{}数据集提供成对的空中图像和对应的深度图合成对齐的真实水下结果。Underwater-GAN~\cite{yu2018underwater}使用基于模型的浑浊度模拟生成网络,以合成不同衰减程度的水下图像。UWCNN~\cite{li2020underwater}使用衰减系数模拟近海岸和海洋里面不同水类型,控制衰减系数来进行多种水类型的图像合成。这些方法生成的结果需要依赖于水下信息,对于真实水下采集到的图像数据缺少泛化能力。

另一种是基于CycleGAN,无监督一对一映射的图像转译方法,一个域是空中图像、另一个域是水下图像,通过循环一致来学习从空中到水下图像的转译。UGAN~
\cite{fabbri2018enhancing}和MLFcGAN~\cite{liu2019mlfcgan}等都是用这种方法进行了水下图像合成,这些方法不依赖于任何水下光学模型和任何物理信息。然而基于CycleGAN的方法都是一对一映射,在图像合成过程中无法生成一对多的多模态结果。


\section{课题来源}
国家自然科学基金面上项目“ 类别不平衡条件下海洋浮游生物图像精细识别及其原位应用研究”(批准号:61771440)、   国家自然科学基金面上项目“ 海洋中小型浮游生物原位光学观测关键技术研究”(批准号:41776113)。


\section{论文内容和安排}
本工作主要内容和章节安排如下:

第一章为绪论部分,该部分主要针对基于生成对抗网络的图像转译和水下图像转译的研究背景和意义、国内外研究现状以及课题来源进行论述。

第二章介绍了生成对抗网络的基本概念、基于生成对抗网络的主流算法和基于生成对抗网络应用于图像转译的方法。详细对算法概念和原理、网络模型结构、生成对抗网络及主流算法优缺点以及在图像转译场景上进行了具体介绍。

第三章介绍了水下图像合成的原理和问题、水下图像评测指标以及水下图像处理方法、基于深度学习的水下图像多样式合成方法。对水下图像以及水下多种样式图像的合成方法进行深入的分析和了解,为后续提出水下图像跨域多模态转译模型和水下图像多样式域转译模型奠定基础。

第四章主要描述了本工作提出的基于生成对抗网络的水下图像跨域多模态转译模型和基于生成对抗网络的水下图像多样式域转译模型设计,其中包括问题定义,问题分析以及模型网络结构设计和目标函数设置。随网络模型设计思路以及具体的模型实现细节做了详细展现。

第五章介绍了水下图像跨域多模态转译模型和水下图像多样式域转译模型的实验和分析评测。对设计的模型进行实验设置,使得模型能够充分展示其效果,并将实验结果进行分析。

第六章对本工作进行贡献总结和讨论,分析研究问题的特点以及提出方法的优点和缺点,对研究过程中固有问题进行分析和展望。