\chapter{绪论}
\section{研究背景和意义}

建设海洋强国是中国特色社会主义事业的重要组成部分。党的十八大报告明确指出,要提高海洋资源开发能力,坚决维护国家海洋权益,建设海洋强国。海洋是生命的摇篮、资源的宝库、交通的命脉、战略的要地。作为海洋研究的重要组成,水下视觉技术在水下机器人设计~\cite{kim2012object,li2020object}、海洋救援任务、海洋生物追踪、实时导航~\cite{chen2012real}等领域中具有广泛应用。水下视觉可以在不具有数据采集条件的条件下为研究者提供必要的技术支持~\cite{ludvigsen2007applications,万媛媛2012水下光视觉目标检测与定位系统关键技术研究},真实水下充满了未知,与陆地视觉研究相比数据采集具有更多的限制。

从生成对抗网络~\cite{goodfellow2014generative}问世开始,计算机视觉领域打开了一个全新的局面,愈来愈多的工作基于生成对抗网络开展。在图像处理领域,可以通过训练简单的基于生成对抗网络的模型合成大量现实中无法获取到但是具有足够真实性的图像数据,其中图像到图像翻译~\cite{isola2017image}被单独划分成一类问题进行解决。图像到图像的翻译简称为图像翻译,图像翻译是指将输入图像“翻译”成一幅相对应的输出图像,输入图像与输出图像之间具有相同或者相类似的内容信息但是具有不同的场景特征。比如翻译成一副对应内容的边缘信息图、语义标签图或者特定场景风格图。我们将这种翻译视为从源域到目标域的映射,输入和输出分别属于两个不同的域。

% 在研究图像翻译问题中,根据可获得的数据集是否包含成对的训练图像可以将解决方法划分为成对的图像翻译子问题和非成对的图像翻译子问题。当数据集满足包含成对的训练图像时,成对的图像翻译问题就可以通过条件生成对抗网络等配对的方法来进行解决。现实情况更多的是不满足获取到大量的成对数据,或者像水下场景等获取成对训练数据特别困难、代价高昂,非成对图像数据的非配对翻译模型被提出。另外,源域和目标域的训练图像数量不平衡,往往一个域中图像数据数量远远超过另一个域图像数据数量,此时给图像翻译带来巨大的难度。

类似我们谷歌翻译一句话,翻译器不会改变我们输入的内容和意思,但是可以翻译成英语、法语等其它语种。在图像翻译中,当目标属于多个种类时,目标域就是多个域,从源域到多个目标域的翻译称为多域翻译~\cite{choi2018stargan,choi2020stargan,lee2020drit++}。谷歌翻译器进行翻译时,不同语种之间的翻译很多情况下并不是一一对应的,一个语义内容可以对应多个翻译结果。在图像翻译中,不成对数据上训练的非配对图像翻译模型可以进行一对多的翻译,网络无法固定学习到特定的映射,可以产生多样输出即多模态翻译~\cite{huang2018multimodal,lee2018diverse,lee2020drit++}。多模态翻译和多域翻译工作是图像翻译的两个重要内容以及经典方向。

图像翻译目前已经有大量的应用,现实中的很多问题都可以转换成图像翻译问题进行解决,图像翻译具有强大的应用价值。可以应用于边框图上色~\cite{ghosh2019interactive}、素描画上色~\cite{cao2017unsupervised,nazeri2018image}等任务,极大的解放了人力;可以应用于图像超分辨重建任务~\cite{ledig2017photo,zhang2019ranksrgan,anokhin2020high},提升图像质量~\cite{wang2019discriminative,bowles2018gan};可以应用于人脸表情漫画化~\cite{yi2019apdrawinggan,shi2019warpgan}、人体姿态图像编辑~\cite{chan2019everybody,cao2019improving}等,实现特定风格的图像生成和控制~\cite{karras2019style},已经落地图像美化等商业软件中;以及在医学图像~\cite{zhou2019prior,lee2019collagan}、水下图像~\cite{uplavikar2019all,li2020underwater}等多种特定学科的常规手段难以直接进行解决的问题中得到应用,给其他学科的图像问题解决提供了新的思路和方法。

在水下环境中,由于采集技术和设备的局限性,整体上在水下环境中的数据采集耗费巨大、成本高昂;另一方面由于水下光学物理特性,光在水介质的传播过程中有吸收和散射,会出现色偏、模糊、有雾状效果等多种样式。然而,对于同一目标却很难获取到大量多样式水下图像数据,许多水下图像处理任务,都需要成对以及多种样式的水下数据。可见,同时获得多种水下环境条件的高质量图像数据对水下图像研究领域意义重大,可以为水下视觉领域提供大量的数据,推动水下研究工作的发展和进步。随着深度学习的广泛发展,生成对抗网络开始越来越多的被应用于水下图像研究,为水下图像处理提供了简洁、有效、不受物理模型限制、具有泛化性的方法。

水下研究中,输入图像与输出图像之间具有相同的内容、不同的场景信息,水下图像合成可称水下图像翻译,在此基础,我们研究基于生成对抗网络的水下图像翻译问题。水下图像翻译与陆地图像翻译不同,会受到许多因素的影响,根据水体光学参数不同,结果呈现出不同的浑浊度和不同的色偏效果。我们提出利用基于生成对抗网络的图像翻译方法处理图像,无需针对多种样式的水下环境考虑水体光学参数和受到物理模型限制。建立在真实水下图像和合成水下图像数据集上,以生成对抗网络作为基础,通过学习水下图像成像特点,将给定的数据转化为多种样式水下数据。在此基础上,还可以探索在指定样式条件下基于生成对抗网络的图像翻译方法进行水下图像翻译。

\section{国内外研究现状}
\subsection{生成对抗网络研究现状}
2014年,在Goodfellow等人~\cite{goodfellow2014generative}提出生成对抗网络(GAN)这项意义深远的工作后,计算机视觉领域的各项工作进入爆发期。生成对抗网络由生成器和判别器组成,生成器试图生成可以欺骗过判别器的真实性结果,判别器负责区分输入的样本是真实样本还是生成器合成的结果。这样处于对抗关系中的两个神经网络彼此不断优化,训练生成器能够生成足够真实的合成结果。

第一个生成对抗网络结构只能应用于相对简单的数据集,这是因为全部使用了全连接神经网络~\cite{goodfellow2014generative}。而卷积神经网络非常适合做图像,Radford等人~\cite{radford2015unsupervised}提出DCGAN,整个模型使用卷积神经网络取代全连接神经网络。CGAN~\cite{mirza2014conditional}将GAN网络拓展至分类条件设置,CGAN对多模态数据生成能够提供更有优势的效果。InfoGAN~\cite{chen2016infogan}将噪声源分解成不可压缩的噪声和潜在编码向量,潜在编码可以用非配对的方式学习目标类。ALI~\cite{donahue2016adversarial}和BiGAN~\cite{zhang2018bidirectional}引入编码器(推理网络),将输入映射到潜在空间向量的推理机制。自动编码器是由编码器和解码器组成网络,编码器从数据空间学习到潜在表示空间的确定性映射,解码器进行从潜在空间到数据空间映射,这两个映射的合成组成重建,并重建结果尽可能接近原始输入。

在研究生成对抗网络时有两个思路,一种是减轻生成对抗网络训练不稳定的问题和模式崩溃问题,另一种是应用于计算机视觉、自然语言处理和其它领域。针对训练不稳定和模式崩溃问题,Mixture Density GAN~\cite{eghbal2019mixture}使用混合密度高斯模型代替原先的模型,分类器除了判别图像的真假外,在嵌入空间中分类器被激励形成多个聚类,判断输入图像属于哪个类型。SRGAN~\cite{liu2019spectral}提出了高斯谱正则化方法,通过对权重矩阵的谱分布进行补偿来防止它们崩溃,进而成功地避免了GAN中的模式崩溃问题。An等人~\cite{an2020ae}提出AE-OT-GAN模型,在生成高质量图像的同时克服了模式崩溃问题。生成对抗网络另一种研究思路是应用于计算机视觉任务,图像合成~\cite{wang2019discriminative}、目标检测~\cite{wang2017fast}、分类和识别~\cite{oza2020multiple,fang2020generate,jung2020icaps}、语义分割~\cite{luc2016semantic}等图像、视频具体任务。

\subsection{图像翻译研究现状}

图像合成包括图像到图像翻译~\cite{isola2017image},文本到图像合成~\cite{zhu2019dm},图像超分辨率~\cite{hyun2020varsr,lee2020journey},去雾去噪~\cite{shao2020domain,wan2020reflection}等。基于生成对抗网络的图像到图像的翻译首次被Isola等人~\cite{isola2017image}在2017年的工作pix2pix中提出,依托条件为图像的条件生成对抗网络思想,训练图像使用尺寸相同的成对数据,是最典型的配对图像翻译方法。同年,Zhu等人~\cite{zhu2017unpaired}提出CycleGAN,基于生成对抗网络提出的循环一致性损失,该方法解决了更普遍没有成对训练数据的非配对图像翻译问题。
% 自此,图像翻译开始被应用于各种具体问题的解决。
DRPAN~\cite{wang2019discriminative}针对图像翻译中的高质量高分辨率合成问题,构建了一个生成对抗网络框架改善合成图像质量,在配对和非配对前沿模型上有效的提升了图像合成质量。

非配对跨域翻译方法中,UNIT~\cite{liu2017unsupervised}将问题定义为实现两个域之间的翻译是满足两个域的联合分布,假设输入到生成器的潜在编码共享,通过编码器将两个域的图像数据编码到潜在空间中。受分解表达学习的影响,MUNIT~\cite{huang2018multimodal}在UNIT提出共享潜在编码的基础上假设潜在编码包括内容编码和风格编码,可以合成内容相同风格不同的多模态结果。DRIT~\cite{lee2018diverse}与MUNIT思路一致,仅在内容编码和风格编码连接方式上有所区别。以上基于分解表示的图像翻译方法广泛应用于风格迁移和多种多模态图像翻译任务中。DSMAP~\cite{chang2020domain}在MUNIT基础上利用特定于域的映射将共享内容空间中的潜在特征重新映射于特定于域的内容空间,这样保证特定于域的内容空间不含有源域的任何信息。

以上方法都是在两个域之间做图像翻译,若要进行多个域之间的翻译需训练多个域间生成器一一对应,StarGAN~\cite{choi2018stargan}用一个生成器结构学习多个域之间的映射从而实现多个域的翻译。StarGAN v2~\cite{choi2020stargan}将StarGAN多域翻译工作拓展至多模态,可以 实现多域多模态的图像翻译任务。在DRIT工作基础上,DRIT++~\cite{lee2020drit++}被提出,除了优化多模态效果外还将两个域之间的翻译也拓展至多个域之间的翻译。


\subsection{水下图像翻译研究现状} 
在研究基于生成对抗网络的水下图像领域时,输入图像与输出图像之间具有相同的内容、不同的场景信息,因此水下图像合成可以称作水下图像翻译。在各种水下图像处理任务中,水下图像数据占据重要地位,有效的水下图像翻译方法研究起着十分关键的作用。水下图像翻译目前主要有以下两种常用的方式,以获得具有真实性的水下图像。

第一种是需要额外的深度图或者传输图等水下信息或基于衰减模型来翻译成具有该特征信息的水下图像翻译方法,主要在室内转水下图像数据集上应用。WaterGAN~\cite{li2017watergan}使用室内Kinect数据集进行翻译,四个RGB-D~\cite{janoch2013category,lai2014unsupervised,silberman2011indoor,shotton2013scene}数据集提供成对的空中图像和对应的深度图翻译成对齐的真实水下结果。Underwater-GAN~\cite{yu2018underwater}使用基于模型的浑浊度模拟生成网络,收集互联网上的图像作为真值,以翻译成17种不同衰减程度的水下图像。UIE-DAL~\cite{uplavikar2019all}和UWCNN~\cite{li2020underwater}使用NYU v2~\cite{silberman2012indoor}数据集中传输图作为衰减系数模拟近海岸和海洋里面不同水类型,控制衰减系数来进行10种水类型的图像翻译。这些方法生成的结果需要依赖于水下信息或者水下,对于真实水下采集到的图像数据缺少泛化能力。

另一种是UGAN~
\cite{fabbri2018enhancing}此类利用CycleGAN这个非配对一对一映射的图像翻译方法,一个域是空中图像、另一个域是水下图像,通过循环一致来学习从空中到水下图像的翻译。MLFCGAN~\cite{liu2019mlfCGAN}和Jamadandi等人~\cite{jamadandi2019exemplar}提出的方法等都是用这种方法进行了水下图像翻译,不依赖于任何水下光学模型和任何物理信息。FUnIE-GAN~\cite{islam2020fast}通过相同的方法拓展成包括配对和不配对子集的大规模EUVP数据集。然而基于CycleGAN的方法都是一对一映射,在图像翻译过程中无法同时生成一对多的多模态结果是此类方法存在的问题。欲得到多种样式结果,需要多个样式对组成的训练数据进行多组训练。

\section{课题来源}
本课题来源于国家自然科学基金面上项目“ 类别不平衡条件下海洋浮游生物图像精细识别及其原位应用研究”(批准号:61771440)和国家自然科学基金面上项目“ 海洋中小型浮游生物原位光学观测关键技术研究”(批准号:41776113)。


\section{论文内容和安排}
本工作主要内容和章节安排如下:

第一章主要针对生成对抗网络,基于生成对抗网络的图像到图像翻译和水下图像翻译的背景意义,国内外研究现状以及课题来源进行介绍。

第二章介绍了生成对抗网络的基本概念、基于生成对抗网络的主流算法和基于生成对抗网络的图像翻译主流方法,引出水下图像翻译的主流方法。概述了算法概念和原理、网络模型结构及算法优缺点等具体内容。

第三章主要描述了本工作提出的基于生成对抗网络的水下图像跨域多模态翻译模型和基于生成对抗网络的水下图像多样式域翻译模型设计,其中包括问题定义和分析、模型网络结构设计和目标函数设置。随网络模型设计思路以及具体的模型实现细节做了详细展现。

第四章介绍了水下图像跨域多模态翻译模型和水下图像多样式域翻译模型的实验和分析评测。对设计的模型进行一系列对比实验和必要的消融实验,充分展示各个模型的实验效果,并将实验结果进行定性和定量分析。

第五章对本工作进行贡献总结和讨论,分析研究问题的特点以及提出方法的优点和缺点,对研究过程中固有问题进行分析和进一步展望。

% 第六章