\chapter{总结与讨论}
\section{全文总结}
本文主要对水下图像的多样式翻译问题进行了深入的探讨,通过对图像翻译工作进行研究,对于图像翻译工作的多模态翻译和多域翻译方向开展实验,尝试通过提出两种不同思路的模型进行解决。本文主要工作如下:

\begin{itemize}
	\item [1.]
	本文对生成对抗网络进行了研究,了解现有基于生成对抗网络的的图像翻译问题的国内外现状,对于生成对抗网络在图像翻译上的思路进行梳理。水下图像数据在水下视觉研究中非常重要,本文调研了水下成像以及水下图像特点,对水下数据和评价进行了解,通过对水下图像翻译方法综述,总结了水下图像多样式翻译的方法以及各自特点。
	
	\item [2.]
	针对水下图像多样式翻译的问题,基于对生成对抗网络上的图像翻译研究,我们梳理了水下图像多样式翻译的两种思路,一种是基于跨域图像多模态翻译方法,将多个水下样式看作水下域的多个模态进行处理,通过引入内容一致性损失限制翻译过程中内容保持完整不损耗;另一种是基于多域翻译方法,将各个样式看作多个域来进行处理,通过样式编码控制到各个域的翻译。并在这两种思路上进行网络模型结构设计和目标函数设置。

	\item [3.]
	对于以上提出的水下图像多样式翻译的两种思路,我们在图像翻译问题中找到最经典和最前沿的方法,对各自思路在多个合成数据集和真实水下数据集上设计了一系列的对比实验和消融实验。我们的评价指标选取了图像翻译中的评价指标和水下图像研究的评价指标,保证在评价过程中足够充分和客观,另外在定性视觉结果和定量指标结果上进行了详细分析,从而验证了我们提出网络结构对于该问题能够有效的解决和其优越性。

\end{itemize}

\section{讨论}
尽管我们提出的两种思路以及对应的两种模型结构都可以出色地解决翻译水下多样式翻译问题,但是目前基准方法和我们提出的基于生成对抗网络的水下图像多样式翻译工作对训练数据有着较高的要求,也是由于基于大量数据的图像翻译方法自身必然会受到数据的限制,在多个不同数据集上,模型训练结果有较大差异,虽然数据自身可以无需配对,但是训练数据中不纯净存或特定图像翻译任务较难时,部分水下图像多样式翻译工作会出现翻译失败的案例。


